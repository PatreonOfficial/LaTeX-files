\documentclass[a4paper]{proc}
\usepackage{graphicx} % Required for inserting images
\usepackage[utf8]{inputenc}
\usepackage[T1]{fontenc}
\usepackage{amssymb}

\title{Test Project}
\author{Justus Osthoff}
\date{August 2024}

\begin{document}
	\section*{Aufgabe 1}
	Wenn $\vec{|x|}$ 20 ergibt, wie groß ist dann x?\vspace{10pt}\\ %Use as deafult task
	A:~5\\
	\\
	B:~\(\frac{20^2}{x}\)\\
	\\
	C:~\(\frac{20}{\sqrt{3}}\) %< This btw
	
	\vspace{20pt} %Use this as default spaceing between tasks
	
	\section*{Aufgabe 2} %Solution is n=4
	Angenommen, wir haben eine Funktion $f(x)=x^n+4$ und die folgende Tabelle, was wäre die Potenz~(Das n)?
	
	\begin{table}[h!]
		\centering
		\begin{tabular}{c||c|c|c|c}
			x & 1 & 2 & 3 & 4\\
			\hline
			y & 5 & 20 & 85 & 260\\
		\end{tabular}
		%\caption{Caption}
		\label{tab:my_label}
	\end{table}
	Die Potenz muss \underline{\hspace{20pt}} sein!
	
\end{document}
